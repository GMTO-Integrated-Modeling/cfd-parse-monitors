\section{Purpose}
\label{sec:purpose}

This document describes the wind loads that the GMTO Mount Element is providing
to the mount vendor.

\section{Introduction}

Following the mount PDR\cite{MOUNT-PDR}, the CFD telescope model has been updated to take
into account the numerous design changes.
The updated CFD model is described in GMT-DOC--05211\cite{GMT.DOC.05211}.

The mount wind loads are detailed in  chapter~\ref{cfd-wind-loads}.
The chapter is divided into 11 sections, one section for each CFD case.
Each case is defined according to a telescope zenith and azimuth angles, a wind
speed and an enclosure configuration.
For the mount wind loads, all the cases correspond to the telescope pointing at a 30 degree zenith angle.
The telescope azimuth is given with respect to the 30$^o$-NNE predominant wind
direction at the GMT site.
The enclosure configuration is either vents open and wind screen stowed (OS) for the
2m/s and 7m/s cases or vents closed and wind screen deployed (CD) for the 12m/s
cases.
Table~\ref{tab:cases} identifies the CFD cases for the mount wind loads.
\begin{table}
  \centering
  \begin{tabular}{c|ccc}\toprule
    Azimuth [degree]  & 2m/s (OS) & 7m/s (OS) & 12m/s (CD) \\\midrule
      0               & \checkmark& \checkmark& \checkmark \\
     45               & \checkmark& \checkmark& \checkmark \\
     90               & \checkmark& \checkmark& \checkmark \\
    135               & & \checkmark&  \\
    180               & & & \checkmark \\\bottomrule
  \end{tabular}
  \caption{CFD settings for the mount wind loads cases.}
  \label{tab:cases}
\end{table}

Each section in  chapter~\ref{cfd-wind-loads} includes a snapshot of the vorticity for the whole Observatory,
two tables providing the mean, the standard deviation, the minimum and the
maximum of the magnitude of either the force or the moment exerted on the
elements of the telescope and time series plots of the force magnitude on the
different telescope elements.
Both, forces and moments are sampled at 20Hz.

The simulations take some time to converge to a stationary solution and this convergence time is a function of the initial conditions but strongly depends on the wind speed and the enclosure configuration.
The duration of the simulations have been set to 1200s, 700s and 900s for the 2m/s, 7m/s and 12m/s CFD runs, respectively.
It is usually the case that the last 400s of the simulations are stationary and
the statistical quantity reported in the forces and moments tables have been derived from
the last 400s of the corresponding time series.
A mistake was made in one report of one C-ring element for 3 cases (Sec.~\ref{zen30az000_OS7}, Sec.~\ref{zen30az045_OS7} and Sec.~\ref{zen30az000_CD12}), the erroneous C-ring report have been corrected and the cases have been run for 400 more seconds. 



%%% Local Variables:
%%% mode: latex
%%% TeX-master: "gmto.cfd2021.mount"
%%% End:
